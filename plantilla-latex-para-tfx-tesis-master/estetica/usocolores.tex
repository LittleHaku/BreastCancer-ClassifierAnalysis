Se puede usar cualquier color usando su nombre o con el comando \textbackslash color\{nombrecolor\} según corresponda. Sin embargo no se recomienda el uso de estos colores de forma aleatoria. Según la gama de colores elegida se puede usar un conjunto de colores mucho más simple y que se corresponden con los de las tablas de color de la gama elegida. Los colores tienen el nombre adecuado al entorno en el que deben ser usados. La lista de estos colores es la siguiente:

\begin{description}
  \item [maincolor] Color principal, usado en casi todos los elementos del documento que necesiten color en el texto.
  \item [dmaincolor] Color principal en su versión oscura (no se usa por defecto en ningún elemento).
  \item [descriptioncolor] Color que se utiliza en elemento descrito en las listas de descripción como esta, por defecto idñentico a `maincolor'.
  \item [headfootcolor] Color utilizado en las cabeceras y pies de página, por defecto idñentico a `maincolor'.
  \item [textcolor] Siempre es negro.
  \item [textboxfgcolor] Color `foreground' de los cuadros de texto.
  \item [textboxbgcolor]  Color `background' de los cuadros de texto.
  \item [codefgcolor] Color `foreground' de los códigos presentados.
  \item [codebgcolor] Color `background' de los códigos presentados.
  \item [equationfgcolor] Color `foreground' de las ecuaciones.
  \item [equationbgcolor] Color `background' de las ecuaciones.
  \item [commentcolor] Color de los comentarios en los códigos.
  \item [complementaryone ... complementaryeight] Son los ocho colores complementarios usables en gráficos.
  \item [complementarylightone ... complementarylighteight] Son los ocho colores complementarios suaves.
  \item [complementarydarkone ... complementarydarkeight] Son los ocho colores complementarios oscuros.
\end{description}
