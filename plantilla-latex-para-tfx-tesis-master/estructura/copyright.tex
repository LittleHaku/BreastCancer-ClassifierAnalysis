El texto del copyright aparecerá siempre por defecto en el reverso de la portada, sin embargo se puede cambiar usando opciones de la clase. Estas opciones pueden tomar tres valores, \textbf{copyright}, \textbf{copyleft} y \textbf{nocopyright}. La primera de ellas muestra un texto de copyright y es el valor por defecto, la segunda un texto de copyleft y la tercera elimina el texto de copyright.

En este tipo de documentos suele hacerse una dedicatoria corta y citar alguna frase celebre. Ambas cosas se se pueden hacer con los comandos \textbackslash dedication\{\} y \textbackslash famouscite\{\} en el preámbulo del documento y que llevan como parámetro la dedicatoria y la cita celebre. En los fuentes de este documento se puede encontrar un ejemplo de estos dos comandos. Debido a cómo son estos comandos no deben introducirse líneas en blanco en los textos y los saltos de línea o párrafo deberán hacerse con \textbackslash\textbackslash\ o \textbackslash\textbackslash[height].
