This thesis aimed to develop a machine-learning model that classified breast cancer fine needle aspiration biopsies as benign or malignant and provided explanations for the model's predictions.

To achieve this goal, different classifiers were built and optimized using the Wisconsin Breast Cancer dataset. The classifiers were then evaluated using a defined set of metrics to identify the most performant one. To gain a deeper understanding of the chosen model's decision-making process, the \ac{shap} library was implemented. This provided both global and local explanations for the model's predictions. The global explanation revealed the overall importance of each feature within the model, while the local explanation pinpointed how individual features contribute to specific predictions.

This model did not have a 100\% accuracy rate, but it was able to classify the majority of the samples correctly, and when it made a mistake it never stated being 100\% sure of the prediction. This is important in the medical field, where the consequences of a wrong prediction can be severe. The model's strength lies in its ability to provide explanations for its predictions, which can help healthcare professionals understand the model's decision-making process and speed up the diagnostic process; ultimately, improving patient outcomes.

The web application developed as part of this project can help other researchers and healthcare professionals understand the models' hyperparameters, metrics, and feature importance. It also allows users to explore a selection of predictions with their respective waterfall plots, which provide a detailed explanation of the model's decision-making process; then they can use this information along with their domain knowledge and the dependence plots to make a more informed decision.

In conclusion, this project has successfully developed a machine-learning model that can classify breast cancer \acl{fnab} samples as benign or malignant and provide explanations for the model's predictions. The model's explanations can help healthcare professionals understand the model's decision-making process and improve the diagnostic process. We encourage the reader to explore the web application (\href{breast-cancer-classification-web.onrender.com}{breast-cancer-classification-web.onrender.com}) or if not possible, check the video demonstration of the application on YouTube (\href{https://youtu.be/ZASpsVNhXPA?si=7OojLfj64Kx2qjkr}{https://youtu.be/ZASpsVNhXPA?si=7OojLfj64Kx2qjkr}). Lastly, the Jupyter Notebook used in this project are available on GitHub \\(\href{https://github.com/LittleHaku/BreastCancer-ClassifierAnalysis/}{https://github.com/LittleHaku/BreastCancer-ClassifierAnalysis/}).