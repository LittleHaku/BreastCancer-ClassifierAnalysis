To display all the results from the classifiers and the \ac{xai} analysis, a web application will be developed. To build the classifiers and the \ac{xai} analysis the programming language Python will be used, and thus for the web application we will continue with this language. This opens the possibility of using Flask or Django as the web framework. Django and Flask both are free and open-source web frameworks for Python, and both are very good choices, but they have different use cases.

Django allows the user to build web applications quickly, it is a high-level framework that includes a lot of features, it is really easy to use a database or validate forms for example, it takes care of a lot of things for the user, and leaves the developer to focus on the application itself. This also means that Django is not that flexible and can be a bit heavy.

Flask, on the other hand, is a micro-framework, it is really lightweight and flexible, and instead of including a lot of features, it allows the user to add the features that are needed, this means that the developer has to take care of a lot of things that Django takes care of, but it also means that the developer has more control over the application. By default, Flask does not include a database or form validation, but it is possible to add these features with libraries \cite{noauthor_foreword_nodate}.

Taking into account that the focus of the project is not to build a web application but to build the classifiers and the \ac{xai} analysis, and that we will need to build a web application that will display the results from the classifiers and the \ac{xai} analysis, which means that we will need to use a database to store the results, we will use Django as the web framework. This will allow us to build the web application quickly and focus on the classifiers and the \ac{xai} analysis.

After choosing the web framework we will need to choose a database, the database will be used to store the different plots and metrics. Django supports a lot of databases, but the most popular ones are PostgreSQL, MySQL, and SQLite. SQLite is the default database for Django, it is a lightweight database that is perfect for development, but it is not recommended for production, so we will not use this one (except for development or local execution). MySQL and PostgreSQL are both good choices. In this project, PostgreSQL will be used since we have worked with it before and it is a robust database perfect for our described scenario.

The backend is already chosen, now we need to choose the frontend. For this, we will go with the standard choice: HTML for the structure, CSS for the style, and JavaScript for the behavior.

To deploy the web application we will use Render\footnote{\url{TODO}}, a cloud platform that will allow us to deploy the web application for free. Even though the free accounts have some limitations, for this project such settings will be enough. For the database hosting, Neon\footnote{\url{TODO}} will be used, since it also has a free tier that will be enough for this project. Lastly, the application will also be able to run locally to not depend on any cloud platform.
