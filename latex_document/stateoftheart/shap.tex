\newacronym{shap}{SHAP}{SHapley Additive exPlanations}

\ac{shap} \cite{lundberg_unified_2017} is a model-agnostic approach to \ac{xai}, that is, it can be used with any model since it only uses the input and output. In short, how \ac{shap} works, is that it uses the Shapley values from cooperative game theory to explain the output of any machine learning model, a concept named after Lloyd Shapley who introduced the idea that got him the Nobel Prize in Economic Sciences in 2012 \cite{roth_shapley_1988}.

The Shapley value is a concept from cooperative game theory, it is a way to fairly distribute the loot obtained by a coalition of players. The difference is that in \ac{shap} instead of players we have features and instead of loot, we have the predicted output as a probability. The Shapley value is the average contribution of a feature to the prediction, and it is calculated by averaging the marginal contributions of a feature to the prediction over all possible orderings of the features. The marginal contribution of a feature to the prediction is the difference between the prediction with the feature and the prediction without the feature. This is done for all possible subsets of features, and then the Shapley value is the average of all these marginal contributions. This is done for all the features, and then we have the Shapley values, which are the importance of the features for the prediction.

To better understand how these values are obtained let's give an example. To change topics instead of breast cancer, let's say we want to predict the price of a car, and for this, we will have three features, the year of the car, the horsepowers, and the brand. Now, we want to know how much each feature contributes to the price of the car. To do this first we have to obtain the \textbf{powerset} of the features, which is the set of all possible subsets of the features, going from $f=0$ to $f=3$, where $f$ is the number of features in the subset, we can see the power set in Figure \ref{FIG:POWER_SET}.

\begin{figure}[Powerset Example]{FIG:POWER_SET}{Powerset with the features of the car example}
    \image{10cm}{}{powerset.png}
\end{figure}

Now that we have all the possible subsets of the features, the next step is to train a model for each subset, the number of models is given by the cardinality of the power set, $2^f$ where $f$ is the number of features, in this case $2^3=8$. All the models are equal, same architecture, same hyperparameters, the only difference is the number of features used to train the model. Now imagine that we have trained all the models, and we have the predictions for all the models, we can see the predictions in Figure \ref{FIG:POWERSET_PREDS}.

\begin{figure}[Powerset With Predictions]{FIG:POWERSET_PREDS}{Example predictions for the powerset of the car features}
    \image{10cm}{}{powerset_preds.png}
\end{figure}

All the nodes that are connected differ by just one feature, knowing this we can see that, for example, the base model, the one with no features, has a prediction of 25.000 euros, and the model with the year of the car has a prediction of 28.578 euros, this means that the year of the car contributes 3.578 euros to the price of the car, and for example, the brand, Honda, contributes -879 euros. This is called marginal contribution and the formula is shown in Equation \ref{EQ:SINGLE_SHAPLEY}.

\begin{equation}[EQ:SINGLE_SHAPLEY]{Single Shapley Value}
    MC_{Brand, \{Brand\}(x)} = Prediction_{\{Brand\}(x)} - Prediction_{\emptyset(x)} = 24.121 - 25.000 = -879
\end{equation}

Each edge also has a weight, the important thing to note is that the weights are the same for all the edges in a row and that the weights in a row add up to 1. The way to obtain this weight then, is by knowing the number of edges in a row and then dividing 1 by the number of edges to obtain the number of edges we can use basic combinatorics. First, the number of nodes in a row is given by the binomial coefficient, which is given by $F \choose f$, where $F$ is the number of features, and $f$ is the number of features in the subset just like in the power set. Then the number of edges is given by $f \cdot {F \choose f}$, that is, to get the weights of the nodes that connect the base model ($f=0$) to the models with one feature ($f=1$) we have to divide 1 by the number of edges, which is $1 \cdot {3 \choose 1} = 3$, and then we get the weights 1/3, 1/3, and 1/3. This is done for all the rows, and we obtain the weights shown in Table \ref{TAB:WEIGHTS}.

\begin{table}[Weights]{TAB:WEIGHTS}{Weights for the example}
    \begin{tabular}{|c|c|c|c|}
        \hline
        $f$ & $F \choose f$ & $f \cdot {F \choose f}$ & Weights \\
        \hline
        0 & 1 & 0 & 0 \\
        1 & 3 & 3 & 1/3\\
        2 & 3 & 6 & 1/6\\
        3 & 1 & 3 & 1/3\\
        \hline
    \end{tabular}
\end{table}

Now that we have the weights, we can do the weighted sum of the marginal contributions, and we obtain the Shapley value, which is the importance of the feature for the prediction. So if we do the weighted sum for the brand, we obtain the result shown in Equation \ref{EQ:SHAPLEY_BRAND}.

\begin{equation}[EQ:SHAPLEY_BRAND]{Shapley Value for the brand}
    \begin{split}
        SV_{Brand}(x) & = \frac{1}{3} \cdot MC_{Brand, \{Brand\}(x)} + \\ 
        & \quad \frac{1}{6} \cdot MC_{Brand, \{Brand, Year\}(x)} + \\ 
        & \quad \frac{1}{6} \cdot MC_{Brand, \{Brand, Horsepowers\}(x)} + \\ 
        & \quad \frac{1}{3} \cdot MC_{Brand, \{Brand, Year, Horsepowers\}(x)} \\
        & = \frac{1}{3} \cdot -879 + \frac{1}{6} \cdot -37259 + \frac{1}{6} \cdot -1075 + \frac{1}{3} \cdot -33357 \\
        & = -17801
    \end{split}
\end{equation}

With this, we can see that the brand contributes -17801 euros to the price of the car, and we can do the same for the other features, we obtain the Shapley values for all the features and see how much each feature contributes to the price of the car, and we can see the results in Table \ref{TAB:SHAPLEY_VALUES}. If we add all the Shapley values we obtain the difference between the base prediction with no features, a lot of times represented as the mean of the target $E(y)$, and the prediction with all the features, which is the prediction of the model, $f(x)$.
\begin{table}[Shapley Values]{TAB:SHAPLEY_VALUES}{Shapley values for the example}
    \begin{tabular}{|c|c|}
        \hline
        Feature & Shapley Value \\
        \hline
        Year & 6948 \\
        Horsepowers & 133507 \\
        Brand & -17801 \\
        \hline
        \textbf{Sum} & \textbf{122654} \\
        \hline
    \end{tabular}
\end{table}

Then we have that the general formula for the Shapley value is given by Equation \ref{EQ:SHAPLEY}. Now we understand how the Shapley values are obtained and what they are, and we can see why \ac{shap} is model-agnostic.

\begin{equation}[EQ:SHAPLEY]{Shapley Value}
   \boxed{ SV_{feature}(x) = \sum_{{set:feature} \in {set}} \frac{|{set}| }{{F \choose |{set}|}} \cdot ({prediction}_{set} - {prediction}_{{set} \backslash {feature}})}
\end{equation}