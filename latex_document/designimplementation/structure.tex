\newacronym{pca}{PCA}{Principal Component Analysis}
%\newacronym{fnab}{FNAB}{Fine Needle Aspiration Biopsy}
\newacronym{eda}{EDA}{Exploratory Data Analysis}

The structure of the project is divided into four main modules corresponding to the main stages of the project: dataset, classifiers, \ac{xai}, and web application. The dataset module is responsible for loading the dataset and splitting 
it into training and testing set.
Then in the classifiers module, we do the \ac{eda}, data preprocessing, and building and optimizing the classifiers. The \ac{xai} module is responsible for the implementation of the SHAP algorithm and the analysis of the results. Finally, the web application module is responsible for the development of the web application. The structure of the project is shown in Figure \ref{FIG:STRUCTURE}. Now we will describe each module in more detail.

\begin{figure}[Structure of the project]{FIG:STRUCTURE}{Structure of the project.}
    \image{}{}{design.png}
\end{figure}

1. \textbf{Dataset module}: This module is responsible for loading the dataset and splitting it into a 70\% training set and a 30\% test set. The dataset used in this project is the Breast Cancer Wisconsin dataset \cite{william_wolberg_breast_1993}, which contains the parameters of the cell nuclei of the sample obtained by a \ac{fnab} of breast masses, an example of a \ac{fnab} is shown in Figure \ref{FIG:FNAB}. The dataset is loaded using the \textit{Pandas} library \cite{team_pandas-devpandas_2020}. The dataset is then split into a 70\% training set and a 30\% test set using \textit{Scikit-learn}. From now on, only the training will be used for the \acl{eda}, data preprocessing, and building and optimizing the classifiers. The test set will \textit{only} be used to evaluate the classifiers. This ensures that we avoid data leakage. 


\begin{figure}[Fine Needle Aspiration Biopsy]{FIG:FNAB}{Fine Needle Aspiration Biopsy \cite{sidey-gibbons_machine_2019}.}
    \image{10cm}{}{fnab_1.png}
\end{figure}

2. \textbf{Classifiers module}: This module is responsible for the \ac{eda}, data preprocessing, and building and optimizing the classifiers. The \ac{eda} is carried out using \textit{matplotlib} and \textit{seaborn} to obtain descriptive and visual statistics of the dataset. The data preprocessing is carried out using \textit{scikit-learn} to scale and normalize the data and to apply \ac{pca}, this is made using pipelines that facilitate the process of building and optimizing the classifiers and their reproducibility. The building and optimization of the classifiers are carried out using \textit{scikit-learn} using the training set to build and optimize the classifiers. Once the classifiers are built and optimized, they are evaluated using the test set.

3. \textbf{\ac{xai} module}: This module is responsible for the implementation of the \ac{shap} algorithm and the analysis of the results. This is implemented using the \textit{shap} library, which is a Python library that allows us to calculate the \ac{shap} values of the classifiers, this values will be used to understand the importance of the features for each classifier and to understand how a prediction was made. This is what makes this project more interpretable than just using classifiers since this will allow a doctor to understand why a prediction was made.

4. \textbf{Web application module}: This module is responsible for the development of the web application. This website was made using \textit{Django} as the backend, \textit{HTML}, \textit{CSS}, and \textit{JavaScript} as the frontend. The website is mainly a way of visualizing the results of the project, it allows the user to see a list of all classifiers and their metrics, which features are the most important for that classifier, and how some predictions were made. It also allows the user to compare two classifiers and see the differences in their metrics and the importance of the features.
