As it has been previously mentioned the dataset used in this project is the Breast Cancer Wisconsin dataset \cite{william_wolberg_breast_1993}. This dataset is composed of 569 samples of breast masses obtained by \ac{fnab}, then these samples were parametrized, meaning that now we will work with numbers instead of images. As a result of this process, we have 30 features that describe the cell nuclei of the sample, these features are the mean, standard error, and 
worst \ab{??? what do you mean?}
of the ten different parameters of the cell nuclei.

Before using these values in the classifier we first need to understand the dataset, this is done using \ac{eda} techniques. This involves, first, obtaining a general understanding of the dataset, then visualizing the dataset, and finally understanding the relationships between the features. All of this is done only using the training set to avoid data leakage.

