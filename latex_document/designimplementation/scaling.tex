This is done using the \textit{scikit-learn} library. The first step is to handle the missing values, this is done using the \textit{SimpleImputer} class, which replaces the missing values with the mean of the feature. Then we scale the features using the \textit{StandardScaler} class, which scales the features to have a mean of 0 and a standard deviation of 1, this is done like the equation \ref{eq:standardization}, where $\mu$ is the mean of all the samples of the feature, $\sigma$ is the standard deviation, and $x$ is the value we want to standardize. We can see in figure \ref{FIG:STANDARDIZATION} how the distribution of the feature stays the same but the mean is 0 and the standard deviation is 1 after standardization.

\begin{figure}[Standardization]{FIG:STANDARDIZATION}{Standardization before and after of mean radius}
    \image{10cm}{}{scaling.png}
\end{figure}

\begin{equation}[eq:standardization]{Standardization}
    \boxed{x' = \frac{x - \mu}{\sigma}}
\end{equation}

