\newacronym{html}{HTML}{HyperText Markup Language}
\newacronym{css}{CSS}{Cascading Style Sheets}
\newacronym{js}{JS}{JavaScript}

To make the results more accessible and visually appealing, we have built a web application using \textit{Django}. This framework was used because \textit{Python} has been the main programming language used in the project and \textit{Django} was the most familiar framework in that language. For the front-end, we used \ac{html}, \ac{css}, and \ac{js}. The web application is hosted in \textit{Render} with the database in \textit{Neon}, which makes it publicly accessible using this \href{https://breast-cancer-classification-web.onrender.com/}{link (https://breast-cancer-classification-web.onrender.com/)}. Since the free hosting of \textit{Render} is not always available, we have also made a video showing the web application in action, this video can be found in this \href{https://www.youtube.com/watch?v=ZASpsVNhXPA}{link (https://www.youtube.com/watch?v=ZASpsVNhXPA)} or if not the project can be run locally by cloning the \href{https://github.com/LittleHaku/breast-cancer-classification-web}{code repository (https://github.com/LittleHaku/breast-cancer-classification-web)}.
The home page of the web application is shown in Figure \ref{FIG:HOMEPAGE}, an example of the comparison of two classifiers that can be prepared using the application is shown in Figure \ref{FIG:COMPARISON}.

\begin{figure}[Web Application]{FIG:HOMEPAGE}{Home page of the web application.}
    \image{}{}{mainpage.png}
\end{figure}

\begin{figure}[Web Application]{FIG:COMPARISON}{Comparison of two classifiers.}
    \image{}{}{comparison.png}
\end{figure}