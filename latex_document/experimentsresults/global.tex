Regarding global analysis two main plots show the importance of each feature in the model. The first plot is the summary plot, which is a bar plot that shows the average \ac{shap} value of each feature, the higher the \ac{shap} value, the more important the feature is. The second plot is the beeswarm plot, which similarly to the summary plot shows the importance of each feature, but also shows if the higher the value of the feature (not the \ac{shap} value, but the actual value of the feature) the higher the probability of the sample being malignant or benign. One of the main advantages of the beeswarm plot is that it shows the distribution of the samples, where each dot in the plot represents a sample, and the color of the dot represents the value of the feature.

\begin{figure}[Beeswarm Plot of Logistic Regression]{FIG:BEESWARMLR}{Beeswarm plot of the Logistic Regression classifier.}
    \image{10cm}{}{lr_beeswarm.png}
\end{figure}

Figure \ref{FIG:BEESWARMLR} shows the beeswarm plot of the \ac{lr} classifier with the features sorted by how much they contributed to the model's decision (the average \ac{shap} value). First, note that the more a dot is to the right, the higher its \ac{shap} value, and thus the higher it contributed to the prediction of the sample being malignant. The same happens if the dot is to the left, but in this case, the sample was predicted to be benign, this explains why there are more dots to the left since we saw in Figure \ref{FIG:ClassDistribution} that there were more benign samples. We can see that the most important features were the worst smoothness, the worst texture, and the worst concave points, then the five following ones are features related to the size of the cells. Also note that 7 out of the 10 most important features are related to the worst values of the features, which means that the worst values of the features are more important for the classifier than the mean or the standard error values.

Lastly, the beeswarm plot shows the distribution of the samples, where the color of the dot represents the value of the feature (see the color bar on the right side of the plot). This allows us to see that in most features the red values are on the right, which means that the higher the value of the feature, the higher the probability of the sample being malignant. There are some exceptions to the rule, like the fractal dimension's features.

After studying the global interpretability of the \ac{lr} classifier, we can see in more detail one feature. To achieve this we can use a plot in which the x-axis represents the value of the feature, and the y-axis represents the \ac{shap} value, there is a line that shows the trend of the feature, if the line is increasing, the higher the value of the feature, the higher the probability of the sample being malignant, and if the line is decreasing, the higher the value of the feature, the higher the probability of the sample being benign. This plot is called the dependence plot. An example of this plot is shown in Figure \ref{FIG:DEPENDENCELR}, on the left side we can see the dependence plot of the worst perimeter which shows that the higher the perimeter, the higher the probability of the sample being malignant, and on the right side we can see the dependence plot of the mean fractal dimension which shows that the higher the fractal dimension, the higher the probability of the sample being benign.

\begin{figure}[Dependence Plots]{FIG:DEPENDENCELR}{Dependence plots of the Worst Perimeter and the Mean Fractal Dimension using the Logistic Regression classifier.}
    \subfigure[]{Dependence plot of the worst perimeter}{
        \image{7cm}{}{worst_perimeter.png}
    }
    \subfigure[]{Dependence plot of the mean fractal dimension}{
        \image{7cm}{}{mean_fractal_dimension.png}
    }
\end{figure}
