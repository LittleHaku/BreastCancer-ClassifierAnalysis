
Breast cancer is the most common cancer type among women \cite{wcrf_international_breast_nodate}; in 2020, there were more than 2.26 million women diagnosed with breast cancer \cite{wcrf_international_breast_nodate}, being the second leading cause of death among women in the United States \cite{american_cancer_society_breast_nodate}. Early detection is a crucial step for improving survival rates. With the current analysis techniques of \ac{fnab},
we have a sensitivity (the ability of a test to identify positive cases correctly) of 0.927 \cite{yu_diagnostic_2012}. Therefore, there is a need for a more accurate interpretation of those tests.

Machine learning is a branch of artificial intelligence that focuses on developing algorithms that can learn from data and extract patterns from it to be then able to generalize it to unseen data. In this case, we can use machine learning classifiers to predict whether a \ac{fnab} sample is benign or malignant.

The potential of classifiers in breast cancer detection is immense. However, the effectiveness of the different classifiers can vary; this is why it is crucial to understand how each classifier works, how to tweak it, and how to make them as precise and effective as possible. Then, once we have the best classifier, we can use \ac{xai} techniques to understand how the classifier works. Then, once we have the best classifier, we can use \ac{xai} techniques to understand how the classifier works so that other types of users, such as, healthcare professionals, or in general, people who are not educated in computer science or mathematics can understand the results of the classifier and trust it. This is crucial for the adoption of machine learning techniques in the healthcare sector, where the results of the classifiers can have a direct impact on the patient's life.

Finding the best possible classifier for this problem would impact cancer detection tasks, facilitating healthcare professionals in their diagnostic responsibilities and, ultimately, improving patient outcomes.