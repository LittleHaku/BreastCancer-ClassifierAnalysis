\ab{the citations are not working, either bibtex was not executed or that information is not included in the bib file}
Breast cancer is the most common cancer type among women \cite{wcrf_international_breast_nodate}; in 2020, there were more than 2.26 million women diagnosed with breast cancer \cite{wcrf_international_breast_nodate}, being the second leading cause of death among women in the United States \cite{american_cancer_society_breast_nodate}. Early detection is a crucial step for improving survival rates. With the current analysis techniques of 
FNAB (Fine Needle Aspiration Biopsy), \ab{not sure if you count with this, but there is a section that could (or should) be included with a list of the acronyms mentioned in the text, try to include as many as possible. There is also an option to make LaTeX do this for you automatically, see commands ac and newacronym}
we have a sensitivity (ability of a test to identify positive cases correctly) of 0.927 \cite{yu_diagnostic_2012}. Therefore, there is a need for a more accurate interpretation of those tests.

Machine learning is a branch of artificial intelligence that focuses on developing algorithms that can learn from data and extract patterns from it to be then able to generalize it to unseen data. In this case, we care about classifiers, whose potential is in the ability to learn from a dataset and then on unseen data being able to classify it as one class or another; in this case, we will be able to classify as benign or malign the results of a fine needle aspiration.

The potential of classifiers in breast cancer detection is immense. However, the effectiveness of the different classifiers can vary; this is why it is crucial to understand how each classifier works, how to tweak it, and how to make them as precise and effective as possible, which is the goal of this thesis.
\ab{idea is good, perhaps a last sentence is missing about how to connect this with XAI and its importance -- maybe this can be mentioned in the next paragraph, where you talk about 'other types of users' (healthcare professionals) who are not mathematicians or computer scientists, who don't need to know the internals of classifiers... that could be a good motivation for XAI, what do you think?}

Finding the best possible classifier for this problem would impact cancer detection tasks, facilitating healthcare professionals in their diagnostic responsibilities and, ultimately, improving patient outcomes.