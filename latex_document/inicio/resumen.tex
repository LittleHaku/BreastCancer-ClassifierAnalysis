Este Trabajo de Fin de Grado
presenta un análisis comparativo de clasificadores para la detección del cáncer de mama y el uso de Inteligencia Artificial Explicable (XAI) para interpretar los resultados. En la fase inicial se realizará la construcción y optimización de los modelos de clasificación, estos clasificadores analizarán los resultados de las biopsias de aguja fina y clasificarán las muestras como benignas o malignas.

Posteriormente, se realiza una comparación de rendimiento comparando métricas como la puntuación F1 o la \textit{recall}. El objetivo es identificar el mejor clasificador de acuerdo a nuestras métricas. Una vez encontrado el mejor modelo, nos adentramos más en él para entender cómo funciona. Para esto, utilizaremos SHAP (SHapley Additive exPlanations), un método de XAI que nos permite ver la importancia de cada característica y cómo contribuyen a la decisión final del modelo. Esto nos permitirá no solo clasificar las muestras, sino también entender por qué el modelo ha tomado esa decisión, lo que puede ser un avance en la comprensión de los modelos de 
Inteligencia Artificial
para fines médicos.

Por último, para mostrar todos los resultados de los clasificadores y el análisis XAI, se desarrollará una aplicación web. Esta aplicación contendrá las métricas resultantes de los clasificadores, así como ejemplos de cómo cada clasificador tomó algunas de sus decisiones y cuáles fueron las características más importantes para la decisión del clasificador.

\palabrasclave{Detección de Cáncer de Mama, Clasificadores, Análisis Comparativo, Interpretabilidad, SHAP, Inteligencia Artificial Explicable, Visualización}
